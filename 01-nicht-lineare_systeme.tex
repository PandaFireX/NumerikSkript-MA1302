%%%%%%%%%%%%%%%%%%%%%%%%%%%%%%
\section{Nichtlineare Systeme}
%%%%%%%%%%%%%%%%%%%%%%%%%%%%%%

Problemstellung: Sei eine stetige Funktion
$F: \Omega \longrightarrow \mathbb{R}^n, \Omega \subset \mathbb{R}^n$ gegeben.
Gesucht ist $x^*\in \Omega$ mit $F(x^*)=0$.
\newline

\begin{bemerkung}
In der Regel existiert keine analytische Formel für $x^*$.
Es existiert kein Algorithmus, der nach endlich vielen Schritten exakt $x^*$
liefert.  Man ist daher auf Näherungsverfahren angewiesen.  Dieses Kapitel
liefert Konstruktionsansätze und untersucht Konvergenzeigenschaften.
\end{bemerkung}

%%%%%%%%%%%%%%%%%%%%%%%%%%%%%%%%%%%%%%%%%%%%%%%%%%%%%%
\subsection{Nullstellenbestimmung im Eindimensionalen}

Zunächst betrachten wir den Spezialfall $n=1$ und die stetige Funktion $f:
\mathbb{R} \longrightarrow \mathbb{R}$, die eine eindeutig bestimmte Nullstelle
$x^*$ besitzt. Geometrische Überlegungen sind die Grundlage für die folgende
Verfahren.


%%%%%%%%%%%%%%%%%%%%%%%%%%%%%%%%%%%
\subsubsection{Bisektionsverfahren}

\begin{tabbing}
\textbf{Algorithmus:} Bisektionsverfahren\\
\= \kill
Input: Startwerte $x_0, x_1, f$ mit $f(x_0) \cdot f(x_1) <0$\\
\newline
\\\> For \ \ \=$k=1,2,3...$\\ 
\> \> $\mathrm{x_{k+1}}= \frac{1}{2} \cdot (x_k + x_{k-1})$ \\
\> \> If \ \ \=$f(x_{k+1})\cdot f(x_{k-1})<0$\\
\> \> \> $x_k=x_{k-1}$ \\
\end{tabbing}

\begin{bemerkung}
Jedes Iterationsverfahren muss mit einem Abbruchkriterium versehen werden.
\end{bemerkung}


\begin{enumerate}
\item[a)] Positiver Abbruch: Die aktuelle Lösung genügt einer vorgegebenen Genauigkeit
      (z.B: $|x_{k+1}-x_k|\leq TOL$).
\item[b)] Negativer Abbruch: Die Verfahrensvorschrift bricht zusammen z.B wegen Division durch 0
      oder es liegt keine oder nur sehr langsame Konvergenz vor. $\rightarrow k_{max}$
\end{enumerate}

\begin{lemma}
Das Bisektionsverfahren ist konvergent (global) unter der Voraussetzung:
$f(x_0) \cdot f(x_1) <0$. Im k-ten Schritt gilt: $|x^*-x_{k+1}|\leq
(\frac{1}{2})^k \cdot |x_1-x_0|$ 
\end{lemma}

\begin{proof}
~
\begin{enumerate}
\item[a)] In jedem Schritt wird das aktuelle Intervall halbiert. 
\item[b)] Die Auswahl des neuen Intervalls erfolgt so, dass $x^*$ darin enthalten ist.
\end{enumerate}
Trotz der Konvergenzaussage findet das Verfahren in der Praxis kaum Verwendung! ($\rightarrow$ langsame Konvergenz, Systemfall)
\end{proof}

%%%%%%%%%%%%%%%%%%%%%%%%%%%%%%%%%
\subsubsection{Sekantenverfahren}

Sekantengleichung durch $(x_k, f(x_k))$ und $(x_{k-1}, f(x_{k-1}))$ gegeben.
\newline
\\$s(x)=\frac{f(x_{k-1})}{x_{k-1}-x_k} \cdot (x-x_k)+\frac{f(x_k)}{(x_k-x_{k-1})}\cdot (x-x_{k-1})$
\newline
\\Berechne $\tilde x$, sodass $s(\tilde x)=0$
\begin{tabbing}

\textbf{Algorithmus:} Sekantenverfahren\\
\= \kill
Input: Startwerte $x_0, x_1, f$\\
\newline
\\\> For \ \ \=$k=1,2,3...$\\ 
\> \> $\mathrm{x_{k+1}}= x_k - \frac{x_k-x_{k-1}}{f(x_k)-f(x_{k-1})} \cdot f(x_k)$ \\
\end{tabbing}

\begin{bemerkung}
Im Gegensatz zum Bisektionsverfahren, kann
\begin{enumerate}
\item[a)] es zum ungewollten Abbruch kommen.
\item[b)] Divergenz auftreten.
\end{enumerate}

Abhilfe: Regula falsi kombiniert die beiden Verfahren, dh.
\begin{enumerate}
\item  $x_{k+1}$ wird aus der Sekantenformel berechnet.
\item  ob mit $(x_{k+1},x_k)$ oder mit $(x_{k+1},x_{k-1})$ weitergerechnet wird, entscheiden die Vorzeichen.
\end{enumerate}
\end{bemerkung}


%%%%%%%%%%%%%%%%%%%%%%%%%%%%%%%%
\subsection{Fixpunktiterationen}

Einen abstrakten Rahmen liefert der Banach'sche Fixpunktsatz aus der Analysis.
\newline

\noindent Sei $\Omega$ eine abgeschlossene Teilmenge des $\mathbb{R}^n$ und $\Phi: \Omega
\longrightarrow \Omega$ eine Kontraktion, dh. $\exists \ q<1: \| \Phi(x) -
\Phi(y)\| \leq q\cdot \|x-y\|$ \ \ $x,y \in \mathbb{R}^n$, so besitzt $\Phi$
genau einen Fixpunkt, dh. ein $x^*$ mit $\Phi(x^*)=x^*$.  Um von dieser Aussage
profitieren zu können, müssen wir zunächst die Problemstellung umformulieren.
Dazu wird zu vorgegebenem $F$ ein $\Phi$ bestimmt, sodass aus $\Phi(x^*)=x^*
\Rightarrow F(x^*)=0$.

\begin{tabbing}
\textbf{Beispiel:}
\\$f(x)=2-x^2-e^x$, gesucht $x^*>0$ mit $f(x^*)=0$\\
\hspace*{8mm} \=$\varphi_1(x)=\sqrt{2-e^x}$\\
\>$\varphi_2(x)=ln(2-x^2)$\\
\end{tabbing}

\begin{definition} [Fixpunktiteration]
Ausgehend von der Iterationsfunktion $\Phi$ und einem Startwert $x_0$ definiert man die Iterationsfolge $\{x_k\}_{k=0}$ durch $x_{k+1}=\Phi(x_k)$.\\
\end{definition}

\begin{lemma}[Konvergenz]
~ % erzwinge einen zeilenumbruch
\begin{enumerate}
\item[a)] Sei $\Phi$ stetig und $x_k \longmapsto x^*$, so gilt $\Phi(x^*)=x^*$\\
\item[b)] Sei $\Phi: \Omega \longrightarrow \Omega$ \ ($\Omega$ abgeschlossen) eine Kontraktion, so konvergiert die Fixpunktiteration.
\end{enumerate}
\end{lemma}

\begin{proof}
~
\begin{enumerate}
\item[a)] $x^*=\lim_{k \to \infty}x_k=\lim_{k \to \infty}\Phi(x_{k-1})=\Phi(\lim_{k \to \infty}x_{k-1})=\Phi(x^*)$
\item[b)]
\begin{align*}
\|x_{k+1}-x_k\| = \|\Phi(x_k)-\Phi(x_{k-1})\| &\leq q\cdot \|x_k-x_{k-1}\| \\
                                              &\leq q^2\cdot \|x_{k-1}-x_{k-2}\| \\
                                   \leq \cdots&\leq q^k\cdot \|x_1-x_0\|
\end{align*}
\end{enumerate}

Daraus folgt, dass $\{x_k\}$ eine Cauchyfolge ist, da
\\$\|x_{k+m}-x_k\|\leq \underbrace{\|x_{k+m}-x_{k+m-1}\|}+\underbrace{\|x_{k+m-1}-x_{k+m-2}\|}+\cdots +\underbrace{\|x_{k+1}-x_k\|}$
\\\hspace*{21mm}$\leq q^{k+m-1}\cdot \|x_1-x_0\|$\hspace*{3mm}$\leq q^{k+m-2}\cdot \|x_1-x_0\|$ \ $\cdots \ \leq q^k\cdot \|x_1-x_0\|$\\
\\\hspace*{21mm}$\leq q^k\|x_1-x_0\| \cdot (1+q+q^2+\cdots +q^{m-1})$\\
\\\hspace*{21mm}$\leq q^k\|x_1-x_0\| \cdot \frac{1}{1-q}$ \ (geometrische Reihe!)
\newline
\\Da $\Omega$ abgeschlossen, konvergiert $x_k$ gegen $x^* \in \Omega$.\\
\end{proof}

\textbf{Beispiel:} Sekantenverfahren
\\\hspace*{8mm}$\Phi(y)=\begin{pmatrix} y_2 \\ y_2-\frac{y_2-y_1}{f(y_2)-f(y_1)}\cdot f(y_2)\end{pmatrix} \ \ y\in \mathbb{R}^2$\\
\newline

\begin{definition}[Konvergenzordnung]
~
\begin{enumerate}
\item Eine Folge $\{x_k \}_{k\in\mathbb{N}}$ heißt konvergent von der Ordnung $p\geq 1$ gegen $x^*$,
falls gilt $\| x_{k+1}-x^*\|\leq C\cdot \|x_k-x^*\|^p$ und für $p=1$ wird gefordert, dass $C<1$ gilt.
($p=1$ linear konvergent, $p=2$ quadratisch konvergent $\cdots$)
\item Der Fall $\lim_{k \to \infty}\frac{\|x_{k+1}-x^*\|}{\|x_k -x^*\|}=0$ heißt superlineare Konvergenz.
\item Eine Fixpunktiteration heißt
  \begin{enumerate}
  \item lokal konvergent von der Ordnung $p$, falls eine offene Umgebung $U$ von $x^*$ existiert,
        sodass $\forall \ x_0\in U$ \ $1_)$ erfüllt ist.
  \item global konvergent von der Ordnung $p$, falls $1_)$ $\forall \ x_0\in \Omega$ erfüllt ist.
  \end{enumerate}
\end{enumerate}
\end{definition}

%\pagebreak
\begin{lemma}[Superlineare Konvergenz des Sekantenverfahrens]
Sei $f$ zweimal stetig differenzierbar und $x^*$ eine einfach Nullstelle, so konvergiert das Sekantenverfahren lokal mit $p=\frac{(1+\sqrt{5})}{2}$ (goldener Schnitt).
\end{lemma}

\begin{proof}
Definiere
\begin{align*}
f[x,y]   &:= \frac{f(y)-f(x)}{y-x} \\
f[x,y,z] &:= \frac{f[y,z]-f[x,y]}{z-x}
\end{align*}

Dann gilt:
\begin{align*}
x_{k+1}&=x_k-\frac{f(x_k)}{f[x_{k-1},x_k]}
\\x_{k+1}-x^*&=(x_k-x^*)\cdot (1-\frac{f[x_k,x^*]}{f[x_{k-1},x_k]})
\\&=\frac{(x_k-x^*)\cdot(x_{k-1}-x^*)}{f[x_{k-1},x_k]}\cdot (f[x_{k-1},x_k,x^*])
\end{align*}

Da $f\in C^2$ folgt aus dem Mittelwertsatz der Differentialrechnung:
\begin{align*}
f[x_{k-1},x_k]     =f'(\eta_k), \ \eta_k\in [\min(x_k,x_{k-1}), \max(x_k,x_{k-1})] \\
f[x_{k-1},x_k,x^*] =\frac{1}{2}\cdot f''(\xi_k), \ \xi_k\in [\min(x_k,x_{k-1},x^*), \max(x_k,x_{k-1},x^*)]\\
\end{align*}

Weiterhin gilt $f'(x^*)\neq 0$ und somit existiert ein $\varepsilon > 0$ und ein $M<\infty$ mit
\[ % alleinstehende formel
  \left| \frac{f''(\xi}{2\cdot f'(\eta)} \right|<M \ \ \forall \ \xi,\eta\in [x^*-\varepsilon,x^*+\varepsilon]=\overline{W}
\]

Sei nun $\delta:= \min (\frac{1}{M},\varepsilon)$ und $U:=(x^*-\delta,x^*+\delta)\subset \overline{W}$
Seien $x_0, x_1\in U$ und $e_k:=M\cdot \|x_k-x^*\|$, so gilt:
\begin{align}
e_k< \min(1,\varepsilon \cdot M)  & \mbox{ für } k>0 \label{eq:prop1}\\
x_k\in U                          & \mbox{ für } k>0 \label{eq:prop2}\\
e_k\leq e_{k-1}\cdot e_{k-2}      & \mbox{ für } k>2 \label{eq:prop3}
\end{align}

Für $k=0,1$ \checkmark\newline

Induktiv erhält man, falls $x_{k-2},x_{k-1}\in U$, dass
\[
  e_k=\frac{1}{M}\cdot e_{k-1}\cdot e_{k-2}\cdot \underset{<M}{\underbrace{\left|\frac{f''(\xi_{k-1})}{2\cdot
  f'(\eta_{k-1})}\right|}} \Rightarrow \mbox{ \eqref{eq:prop1}, \eqref{eq:prop3}}
\]
\[
  \|x_k-x^*\|=\frac{e_k}{M}<\min(\frac{1}{M},\varepsilon)=\delta\Rightarrow
x_k\in U\Rightarrow \mbox{ \eqref{eq:prop2} }
\]
Da $e_0,e_1<1$ gilt $L:=\max(e_0,e_1^\frac{1}{p})<1$. Mehr noch: $e_0\leq L$,
$e_1\leq L^p$. Induktiv ergibt sich daraus $e_k\leq L^{p^k}$ für $k>0$,
da
\[
  e_k\leq e_{k-1}\cdot e_{k-2}\leq L^{p^{k-1}}\cdot L^{p^{k-2}}=L^{p^{k-2}\cdot (1+p)}=L^{p^k}\Rightarrow e_k\leq (L^{p^{k-1}})^p
\]
\end{proof}
